\section{Simulation of vessel motion in waves} 
\label{Simulation of vessel motion in waves}

This section describes{ simulation of vessel motion in waves. The first step is
to generate all the waves and this is described in section \ref{Simulation of
ocean waves}. After generating the ocean waves the vessel response motion is
calculated for each wave for each heading angle and speed. The results are then
saved in a series of tables. 

\subsection{Calculating motion response for each wave}
\label{Calculating motion response for each wave}

\begin{algorithmic}
  \FOR {each regular wave in the simulation field} 
    \FOR {each heading angle between 0 and 360 degree}
      \FOR {each vessel speed} 
        \STATE calculate encounter frequency 
        \STATE calculate total pressure by integrating pressure below the wetted
        surface area 
        \STATE calculate total added mass by integrating added mass along the
        length of ASV 
        \STATE calculate total damping coefficient by integrating damming 
        coefficient along the length of ASV 
        \STATE calculate hydrostatic stiffness 
        \STATE calculate response amplitude 
        \STATE calculate response motion phase lag
      \ENDFOR
    \ENDFOR
  \ENDFOR
\end{algorithmic}

\subsection{Calculating total pressure force}
\label{Calculating total pressure force}
The total pressure force is calculated by integrating the pressure along the
wetted surface area. The wetted surface area can be found by finding the
intersection of the hull surface with the wave profile. 

\begin{algorithmic}
  \STATE let $totalPressureForce = 0$ 
  \FOR {each element in the ASV hull mesh}
    \STATE find if the mesh is above, below or intersecting the wave profile.
    \IF{mesh above wave profile}
      \STATE continue
    \ELSIF{mesh below wave profile}
      \STATE get mesh area and centroid 
    \ELSE
      \STATE find intersection of mesh with wave profile
      \STATE find area and centroid for mesh area below the wave profile.
    \ENDIF
    \STATE calculate the pressure force using formula \ref{eq: Froude-Krylov 
    force}. The pressure force should use the centroid of the the patch for 
    coordinate location and area of the patch to get the force.
    \STATE $totalPressureForce += pressureForceOnMesh $ 
  \ENDFOR
\end{algorithmic}

\subsection{Calculating total added mass and total damping coefficient}
\label{Calculating total added mass}
The total added mass and total damping coefficient is calculated using strip
theory. The full length of the ASV will be divided longitudinally into strips
and the added mass and damping coefficients will be calculated for each strip.
Sum of added mass of all strips will give the total added mass of the vessel and
sum of damping coefficients of all strips will give the total damping
coefficient.

\subsection{Calculating response amplitude}
\label{Calculating response amplitude}
Response amplitude is calculated using equation \ref{eq: equation of motion} but instead of solving the equation of motion for each degree of freedom separately, we will use matrices to simultaneously solve the equations of motion for all degrees of freedom. 

$M$ : Mass matrix
\[
\begin{bmatrix}
  $\Delta$ & 0 & 0 & 0 & $+\Delta z_c$ & 0 \\
  0 & $\Delta$ & 0 & $-\Delta z_c$ & 0 & $+\Delta x_c$ \\
  0 & 0 & $\Delta$ & 0 & $-\Delta x_c$ & 0 \\
  0 & $-\Delta z_c$ & 0 & $I_{44}$ & 0 & $-I_{46}$ \\
  $+\Delta z_c$ & 0 & $-\Delta x_c$ & 0 & $I_{55}$ & 0 \\
  0 & $+\Delta x_c$ & 0 & $-I_{46}$ & 0 & $I_{66}$
\end{bmatrix} 
\]
where: \\
$\Delta$ : is total mass of vessel \\
$I_{44}, I_{55}, I_{66}$ : are moments of inertia around the longitudinal, transverse and vertical axis of the ASV. \\
$I_{46} = I_{64}$ : is roll - yaw product of inertia.\\
$x_c, z_c$ : are the coordinates of centre of gravity of the ASV. The ASV is assumed to be transversely symmetric and hence $y_c = 0$

$C$ : Damping matrix
\[
\begin{bmatrix}
\end{bmatrix} 
\]

$K$ : Stiffness matrix
\[
\begin{bmatrix}
\end{bmatrix} 
\]

$F$ : Force vector
\[
\begin{bmatrix}
\end{bmatrix} 
\]

$x$ : Displacement vector
\[
\begin{bmatrix}
\end{bmatrix} 
\]