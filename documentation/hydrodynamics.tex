\section{Hydrodynamics} \label{Hydrodynamics}
This section presents a summary of the theory of ASV response motion in waves
and and is mainly based on \cite{lewis1988principles},
\cite{bhattacharyya1978dynamics} and \cite{barltrop2013dynamics}.

ASV in waves experiences oscillatory motions and these motions have six degrees
of freedom that is three transitional components - surge (in longitudinal
direction, x), sway (in transverse direction, y) and heave (in vertical
direction, z), and three angular components - roll (about the longitudinal axis,
x), pitch (about the transverse axis, y) and yaw (about the vertical axis, z).
Of the six motions, only three are purely oscillatory in nature- heave, pitch
and roll. This is because these motions causes a change in the equilibrium
between gravitational force and buoyancy force acting on the vessel resulting in
a restoring force which brings the vessel back to the equilibrium position.
Surge, sway and yaw does not produce a restoring force and hence these motions
are not oscillatory in nature unless the exciting force itself changes direction
and brings it back to the initial state.

At first, this section explores motion in each degrees of freedom independently
of others. That is, for example, it is assumed that the heave motion is not
affect by and does not affect any other motion. In reality this is not true.
Since the ASV is longitudinally asymmetric, the heave motion will also induce
pitching motion. The relationship between each motion in each degrees of freedom
is explored in the section \todo[inline]{To do: Link to section for coupled motion}.

\subsection{Equation of motion}

The equation of motion is based on Newton's second law of motion. For each
transitional motion component, this means that the force acting on the vessel
should be equal the product of mass and acceleration in that direction and for
each angular motion components it means that the moment acting on the vessel
equals the product of mass moment of inertia and angular acceleration. The
general equation of motion for forced oscillation is:
\begin{equation}
  M\ddot{x} + C\dot{x} + Kx = F_0 \cos{\omega t}
  \label{eq: equation of motion}
\end{equation}
where, $x$ is the displacement (Note: $x$ here referrers to displacement in any
particular direction and not just along the $X$-coordinate.). $Kx$ is the
restoring force and $K$ is the hydrostatic stiffness. A body oscillating in a
viscous medium will experience damping force due to dissipation of energy to the
surrounding medium. The term $C\dot{x}$ referrers to the damping force
experienced by the body. $C$ is the damping coefficient and $\dot{x}$ is the
velocity. $M\ddot{x}$ is the inertia force where $M$ is the mass and $\ddot{x}$
is the acceleration. $F_0$ is the magnitude of the periodic force that act on
the body with a frequency $\omege$.

It is more convenient to represent the cyclic force term using complex numbers
as shown below: 
\begin{equation}
  F_0(\omega, t) = F_0 \cos{\omega t} + i F_0 \sin{\omega t} = 
  F_0 e^{i \omega t}
  \label{eq: force for simple harmoic motion}
\end{equation}
The observed value of force at any instant of time is the real part of $F_0$. It
is assumed that the wave excitation forces and the resultant oscillatory motions
are linear and hormonic. Also the frequency of the motion is assumed to be same
as the frequency of wave encounter but with phase lag of $\phi$. Therefore at
any instant of time $t$, the displacement for an encounter wave frequency of
$\omega$ is:
\begin{equation}
  x(\omega, \phi, t) 
  = x_0 \cos(\omega t - \phi) + i x_0 \sin(\omega t - \phi) 
  = x_0 e^{i (\omega t - \phi)}
  \label{eq: displacement for simple harmonic motion}
\end{equation}
\begin{equation}
  \dot{x} = i \omega x_0 e^{i(\omega t - \phi)} = i \omega x
  \label{eq: velocity for simple harmonic motion}
\end{equation}
\begin{equation}
  \ddot{x} = i^2 \omega^2 x_0 e^{i(\omega t - \phi)} = - \omega^2 x 
  \label{eq: acceleration for simple harmonic motion}
\end{equation}
Applying equations \ref{eq: force for simple harmoic motion}, \ref{eq:
displacement for simple harmonic motion}, \ref{eq: velocity for simple harmonic
motion} and \ref{eq: acceleration for simple harmonic motion} in \ref{eq:
equation of motion}, we get:
\begin{equation}
  -M \omega^2 x + i C \omega x + Kx = F_0 e^{i \omega t}
\end{equation}
Therefore:
\begin{equation}
  (-M \omega^2 + i C \omega + K)x = F_0 e^{i \omega t}
\end{equation}
\begin{equation}
  x = \frac{F_0 e^{i \omega t}}{-M \omega^2 + i C \omega + K}
\end{equation}
Or:
\begin{equation}
  x = H F_0 e^(i \omega t)
\end{equation}
where: 
\begin{equation}
  H = \frac{1}{-M \omega^2 + i C \omega + K}
  \label{eq: complex transfer function}
\end{equation}
$H$ is called the \textit{complex transfer function}, because it transfers input
force to output deflection and maintains phase information. The phase lag:
\begin{equation}  
  \phi = \tan^{-1} \bigg( \frac{C \omega}{K - M \omega^2} \bigg)
  \label{eq: phase lag for simple harmonic motion}
\end{equation}  
and amplitude:
\begin{equation}
  x_0 = \frac{F_0}{\sqrt{(K - M \omega^2)^2 + (C \omega)^2}}
\end{equation}

A key point to note is that when the ASV is moving at velocity of $U_0$ at an
angle $\mu$ with respect to wave then the frequency of oscillation will shift
from the wave frequency to the encountered wave frequency. In this case the
$\omega$ term in the above equations should be replaced by encountered wave
frequency,$\omega_e$:
\begin{equation}
  \omega_e = \omega_0 - \frac{\omega_0^2}{g} U_0 \cos \mu 
  \label{eq: encountered wave frequency }  
\end{equation}

\subsection{Exciting force and moment}
\label{Exciting force and moment}

Since the focus of this section is vessel response motion due to waves, the
forces and moments considered are only the fluid forces and moments due to waves
acting on the vessel. Fluid forces and moments can be subdivided into two types
- Froude-Krylov and diffraction excitation. Froude-Krylov excitation force and
moment is obtained by integrating the pressure due to wave on the wetted surface
area of the hull. Froude-Krylov does not consider the effects vessel on the
incident wave. On the other hand, diffraction excitation are forces and moments
due to modification of incident wave due to the presence of the vessel. In cases
where the length of the incident wave is longer than the vessel length, the
diffraction excitation forces and moments will be of small magnitude and hence
is often ignored. 

Froude-Krylov force it obtained by integrating the pressure due to wave along
the wetted surface of the hull. For a regular sea wave, equation \ref{eq:
pressure head variation wrt frequency deep water} in section \ref{Regular sea
waves} provides the formula for pressure head variation due to the wave. The
Froude-Krylov force on the hull can then be obtained as given below:
\begin{equation}
  \int_{\frac{-L}{2}}^{\frac{L}{2}} 
  \int_{\frac{-B}{2}}^{\frac{B}{2}} 
  \rho g \zeta_a e^{k z} \cos (k x - \omega t) 
  \,dx \,dy 
\end{equation}
Since we consider each component motion separately, instead of considering the
total pressure force, we consider the component of the pressure force acting in
component direction. So for hear the above formula will be modified only to take
the vertical component of the pressure force. So, if the surface $\,dx \,dy$ has
an angle $\phi$ with the vertical plane, them the formula for Froude-Krylov
force for heave is: 
\begin{equation}
  \int_{\frac{-L}{2}}^{\frac{L}{2}} 
  \int_{\frac{-B}{2}}^{\frac{B}{2}} 
  \rho g \zeta_a e^{k z} \cos (k x - \omega t) \cos \phi 
  \,dx \,dy 
\end{equation}  
Similarly, for surge and sway motion (which are not harmonic motion) the
Froude-Krylov force can be obtained by taking the component in each direction.

The exciting moments can similarly be calculated by taking the moments of the forces about the axis of rotations. For pitch, the exciting moment about the transverse axis of the ASV is:
\begin{equation}
  \int_{\frac{-L}{2}}^{\frac{L}{2}} 
  \int_{\frac{-B}{2}}^{\frac{B}{2}} 
  \rho g \zeta_a e^{k z} \cos (k x - \omega t) \cos \phi x
  \,dx \,dy 
\end{equation}
The moments for roll and yaw can also be calculated similarly.
