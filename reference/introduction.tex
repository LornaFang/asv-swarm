\section{Introduction} 

The marine environment in which an autonomous surface vehicle (ASV) operates is
highly dynamic in nature. An ASV in sea is subjected to forces due to wave, wind
and surface current. These forces vary with time in intensity and direction and
affects the motion of the ASV. A simulator that simulates the motion of an ASV
should be able to capture the dynamics of marine environment and its influence
on the ASV's motion.

The commonly used method for ASV motion simulation is to analyse the dynamic
forces acting on the ASV and the related response motion of the vessel for each
instant of time (or simulation step). This is often referred to as time domain
analysis of response motion. The irregular wave exerts a force on the ASV and
this force is estimated by integrating the wave pressure along the wetted
surface of the hull.  The wetted surface is the hull surface below the water
surface (or the wave profile) and is obtained by finding the intersection of the
hull geometry with the geometry representing the wave profile. The response
motion for each time step is then obtained by solving the equations of motion of
rigid body dynamics. 

Time domain analysis can be very resource intensive since the calculations for
finding the intersection of the 3D hull geometry with the irregular wave
geometry and the integration of the wave pressure force along the wetted hull
surface are to be performed for each time step. This is a disadvantage in cases
where high real time to simulation time ratio is required.

An alternative approach is to use frequency domain analysis. Like in time domain
analysis, in frequency domain analysis the wave forces acting on the vessel is
found by integrating the wave pressure force along the wetted hull surface.
However, frequency domain analysis make two key assumptions regarding nature of
the irregular sea wave and the nature of vessel response motion to the sea wave.
It is considered that the irregular sea is a result of superposition of many
regular waves and the vessel response motion is considered as the resultant of
the vessel's response to each individual regular wave component. It is also
assumed that the vessel response motion varies linearly to the amplitude of the
exciting wave force. These two assumptions makes it possible to split the entire
calculation procedure into two parts. The first part is to create a reference
table of vessel response motion to regular waves of unit amplitude and varying
frequencies. In the second part the irregular wave is broken down into regular
component waves and the vessel response to each of the component wave is
obtained by scaling the response from the table in the first part. The actual
vessel response is then obtained as a resultant of the motion response to
individual wave component.

Frequency domain analysis has some limitations. It is only suitable in the case
where the sea surface is linear superposition of regular waves and vessel motion
varies linearly to wave amplitude.  \todo[inline]{Add example cases where
frequency domain analysis is not suitable.}

The following sections, which are based on references
\cite{lewis1988principles},
\cite{hughes2010ship} and
\cite{bhattacharyya1978dynamics}, 
gradually unfolds the theoretical concepts used in motion simulation using
frequency domain analysis. The first section presents the basic concepts of a
regular sea wave and ASV's motion as a response to the regular sea wave. The
subsequent section then develops on these ideas to present concepts for
representing an irregular sea. This is then followed by a section that presents
the procedure for calculating vessel response motion in irregular sea and the
final section provides a flow chart describing the implementation of frequency
domain analysis in simulation of ASV motion.


