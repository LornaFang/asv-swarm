\section{Introduction} 

The marine environment in which an autonomous surface vehicle (ASV) operates is
highly dynamic in nature. An ASV in sea is subjected to forces due to wave, wind
and surface current. These forces vary with time in intensity and direction and
affects the motion of the ASV. A simulator that simulates the motion of an ASV
should be able to capture the dynamics of marine environment and its influence
on the ASV's motion.

The commonly used method for ASV motion simulation is to analyse the dynamic
forces acting on the ASV and the related response motion of the vessel for each
instant of time (or simulation step). This is often referred to as time domain
analysis of response motion. The irregular wave exerts a force on the ASV and
this force is estimated by integrating the wave pressure along the wetted
surface of the hull.  The wetted surface is the hull surface below the water
surface (or the wave profile) and is obtained by finding the intersection of the
hull geometry with the geometry representing the wave profile. The response
motion for each time step is then obtained by solving the equations of motion of
rigid body dynamics. 

Time domain analysis can be very resource intensive since the calculations for
finding the intersection of the 3D hull geometry with the irregular wave
geometry and the integration of the wave pressure force along the wetted hull
surface are to be performed for each time step. This is a disadvantage in cases
where high real time to simulation time ratio is required.

An alternative approach is to use frequency domain analysis. Like in time domain
analysis, in frequency domain analysis the wave forces acting on the vessel is
found by integrating the wave pressure force along the wetted hull surface.
However, frequency domain analysis make two key assumptions regarding nature of
the irregular sea wave and the nature of vessel response motion to the sea wave.
It is considered that the irregular sea is a result of superposition of many
regular waves and the vessel response motion is considered as the resultant of
the vessel's response to each individual regular wave component. It is also
assumed that the vessel response motion varies linearly to the amplitude of the
exciting wave force. These two assumptions makes it possible to split the entire
calculation procedure into two parts. The first part is to create a reference
table of vessel response motion to regular waves of unit amplitude and varying
frequencies. In the second part the irregular wave is broken down into regular
component waves and the vessel response to each of the component wave is
obtained by scaling the response from the table in the first part. The actual
vessel response is then obtained as a resultant of the motion response to
individual wave component.

Frequency domain analysis has some limitations. It is only suitable in the case
where the sea surface is linear superposition of regular waves and vessel motion
varies linearly to wave amplitude.  \todo[inline]{Add example cases where
frequency domain analysis is not suitable.}

The following sections, which are based on references
\cite{lewis1988principles},
\cite{hughes2010ship} and
\cite{bhattacharyya1978dynamics}, 
gradually unfolds the theoretical concepts used in motion simulation using
frequency domain analysis. The first section presents the basic concepts of a
regular sea wave and ASV's motion as a response to the regular sea wave. The
subsequent section then develops on these ideas to present concepts for
representing an irregular sea. This is then followed by a section that presents
the procedure for calculating vessel response motion in irregular sea and the
final section provides a flow chart describing the implementation of frequency
domain analysis in simulation of ASV motion.

\section{Ocean Waves} 

The outstanding visible characteristic of an open sea surface is its 
irregularity. The waves on the surface do not repeat periodically in time or 
space. Yet, over a wide area and for a period of time, the sea surface maintains
a characteristic appearance. Study on wave data have shown that even though the
sea surface is irregular, the wave elevation is Gaussian in nature and is 
statistically a constant for a given area for a certain period of time. 

For the purpose of this research we consider waves generated due to storms, that
is waves that are generated by the interaction of wind and water surface. The
two main physical process involved in the generation of storm waves are the
friction between air and water and the local variation of pressure field due to
wind. Even though there are many processes that will affect the growth and
propagation of waves, for waves of small amplitude, it is primarily governed by
the principle of superposition. So if $\zeta_1(x,y,t)$ and $\zeta_2(x,y,t)$ are
two wave systems then $\zeta_1(x,y,t) + \zeta_2(x,y,t)$ is also a wave system.
Based on this assumption a sea surface can be defined as a linear superposition
of waves having different amplitude, frequency, wave length and direction. 
However, it should be noted that the assumption regarding the linear 
superimposability of waves fails when the wave system is too steep and wave 
breaking occurs.

\subsection{2D regular sea waves}

A regular sea wave is a harmonic waves with crests that are infinitely long, 
parallel and equally spaced and having constant wave heights. Such waves are
considered as two dimensional because the variation in surface elevation are
only along two coordinates, $x$ coordinate and $z$ coordinate, and is a constant
along the $y$ coordinate. The wave system travels perpendicular to the line of
crests with a velocity $V_c$. It is assumed that water is
incompressible and has zero viscosity. Based on these assumptions the motion of 
water particle in the wave can be described using a quantity called *velocity
potential* which is defined as a function whose negative derivative in any
direction yields the velocity component of the fluid in that direction. Given
below is a simplified equation for velocity potential:
\begin{equation}
  \phi = - \zeta_a V_c \frac{\cosh k(z+h)}{\sinh kh} \sin k(x - V_c t)
  \label {eq: 2D wave velocity potential}
\end{equation}
Where: \\ 
$\zeta_a$ is the wave amplitude at the water surface\\ 
$k = \frac{2 \pi}{L_w}$; $L_w$ is the wave length\\ 
$z$ is the vertical distance of the water particle from the surface and is 
measured negative in the downward direction\\ 
$h$ is the water depth (distance from sea surface to seabed)\\ 
$V_c$ is the wave velocity (or celerity)\\ 
$t$ is time\\ 
$x$ is the x-coordinate of the water particle\\
Figure \ref{fig: RegularWave-1} shows the propagation of wave defined in 
equation \ref{eq: 2D wave velocity potential}.
\begin{figure} 
  \missingfigure{Bhattacharyya Figure 3.13} 
  \caption{Propagation of 2D regular sea wave} 
  \label{fig: RegularWave-1} 
\end{figure}

For deep water, ie. where $h \gg \frac{L_w}{2}$, 
\begin{equation}
  \frac{\cosh k(z + h)}{\sinh kh} \approx e^{kz}
  \label{eq: ratio approx for deep water}
\end{equation}

Substituting equation \ref{eq: ratio approx for deep water} in equation 
\ref{eq: 2D wave velocity potential} we get:
\begin{equation}
  \phi = - \zeta_a V_c e^{k z} \sin k(x - V_c t)
  \label{eq: 2D wave velocity potential for deep water}
\end{equation}

The wave causes variation in the distribution of pressure below the water
surface and the equation for variation of pressure head due to waves is:
\begin{equation}
  \zeta = \frac{k \zeta_a {V_c}^2}{g} \frac{\cosh k(z + h)}{\sinh kh} \cos k(x
  - V_c t)
  \label{eq: pressure head variation}
\end{equation}

For deep water the above equation can be approximated as:
\begin{equation}
  \zeta = \zeta_a e^{k z} \cos k(x - V_c t)
  \label{eq: pressure head variation for deep water}
\end{equation}

For simple harmonic motion: 
\begin{equation}
  \omega = \frac{2 \pi}{T_w} = k V_c
  \label{eq: simple harmonic motion}
\end{equation}
where: \\
$T_w$ is the time period of the simple harmonic wave\\
$\omega$ is the circular frequency of the simple harmonic wave\\

Substituting equation \ref{eq: simple harmonic motion} in 
\ref{eq: 2D wave velocity potential} and 
\ref{eq: pressure head variation}:
\begin{equation}
  \phi = - \zeta_a V_c \frac{\cosh k(z+h)}{\sinh kh} \sin (kx - \omega t)
  \label {eq: 2D wave velocity potential wrt frequency}
\end{equation}
\begin{equation}
  \zeta = \frac{k \zeta_a {V_c}^2}{g} \frac{\cosh k(z + h)}{\sinh kh} \cos (kx
  - \omega t)
  \label{eq: pressure head variation wrt frequency}
\end{equation}

The pressure at any point is given by:
\begin{equation}
  p = \rho g (-z + \zeta)
\end{equation}
\begin{equation}
  p = - \rho g z + \frac{k \zeta_a {V_c}^2}{g} \frac{\cosh k(z + h)}{\sinh kh} 
  \cos (kx - \omega t)
  \label{eq: pressure variation}
\end{equation}

Energy in waves - PNA page 8 top-left 1st para.

\section{Motion in regular sea waves}

\section{Irregular sea}

\section{Motion in irregular sea waves}

\section{Simulation of motion in irregular sea}


