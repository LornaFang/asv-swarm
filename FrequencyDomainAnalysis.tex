\section{Introduction}
The marine environment in which an autonomous surface vehicle (ASV) operates is 
highly dynamic in nature. An ASV in sea is subjected to forces due to wave, wind 
and surface current. These forces vary with time in intensity and direction and 
affects the motion of the ASV. A simulator that simulates the motion of an ASV 
should be able to capture the dynamics of marine environment and its influence 
on the ASV's motion.

There are two approaches for estimating the response motion of a body in waves - 
time domain analysis, and frequency domain analysis. In time domain analysis the 
force exerted by wave on the hull and the corresponding vessel response motion 
is calculated for each instance of time (or simulation step). The wave force is
estimated by integrating the wave pressure along the wetted surface of the hull.
The wetted surface is the hull surface below the water surface (or the wave
profile) and is obtained by finding the intersection of the hull geometry with
the geometry representing the wave profile. The response motion for each time
step is then obtained by solving the equations of motion of rigid body dynamics. 

In frequency domain analysis, unlike time domain analysis, there are two key
assumptions regarding the wave and the vessel response to the wave. The 
irregular wave is considered as a result of superposition of many regular waves
and the vessel response motion is considered as the resultant of the vessel's 
response to each individual regular component of the exciting force. It is also 
assumed that the vessel response motion varies linearly to the amplitude of the 
exciting wave force. These two assumptions makes it possible to split the entire
calculation procedure into two parts. The first part is to create a reference 
table of vessel response motion to regular waves of unit amplitude and varying
frequencies. In the second part the irregular wave is broken down into regular
component waves and the vessel response to each of the component wave is 
obtained by scaling the response from the table in the first part. The actual
vessel response is then obtained as resultant of the motion response to
individual wave component.

Time domain analysis is the most commonly used method for simulating vessel 
motion in waves but this method is resource intensive since the calculations for
finding the intersection of the 3D hull geometry with the irregular wave 
geometry and the integration of the wave pressure force along the wetted hull 
surface are to be performed for each time step. This is a disadvantage in cases 
where high real time to simulation time ratio is required. Frequency domain 
analysis, on the other hand, has an advantage with respect to simulation speed. 
This is mainly due to the fact that the complex, resource intensive part of 
calculating vessel response to regular wave components needs to be performed 
only once for a vessel and the data can be saved and referred during actual 
simulation time.   

Frequency domain analysis has some limitations. It is only suitable in the case
where the sea surface is linear superposition of regular waves and vessel motion
varies linearly to wave amplitude. 
\todo[inline]{Add example cases where frequency domain analysis is not suitable.}

The following sections gradually unfolds the theoretical concepts used in motion
simulation using frequency domain analysis. The first section presents the basic 
concepts of a regular sea wave and ASV's motion as a response to the regular sea 
wave. The subsequent section then develops on these ideas to present concepts 
for representing an irregular sea. This is then followed by a section that
presents the procedure for calculating vessel response motion in irregular sea
and the final section provides a flow chart describing the implementation of
frequency domain analysis in simulation of ASV motion.

\section{Regular sea waves}
A regular sea wave is a harmonic waves having sinusoidal shape with crests that 
are infinitely long, parallel and equally spaced and having constant wave 
heights. Motion of a water particle in the wave can be described using:

\begin{equation} 
  \zeta = \zeta_a \frac{\cosh k(-z + h)}{\cosh k h} \cos k(x - V_w t)
  \label{eq: RegularWave_1}
\end{equation}
where: \\
$\zeta_a$ is the wave amplitude at the water surface\\
$k = \frac{2 \pi}{L_w}$; $L_w$ is the wave length\\
$z$ is the vertical distance of the water particle from the surface and is 
measured positive in the downward direction\\
$h$ is the water depth\\
$V_w$ is the wave velocity (or celerity)\\
$t$ is time\\
Figure \ref{fig: RegularWave_1} shows the propagation of wave defined in
equation \ref{eq: RegularWave_1}.

\begin{figure} 
  \missingfigure{Bhattacharyya Figure 3.13}
  \caption{Propagation of a sine wave}
  \label{fig: RegularWave_1}
\end{figure}

The wave causes variation of hydrostatic pressure and the equation for
hydrostatic pressure in wave at any depth is:

\begin{equation}
  p = \rho g z - \zeta_a \rho g \frac{\cosh k(-z + h)}{\cosh kh} \cos k(x -
  V_wt)
  \label{eq: RegularWave_2}
\end{equation}

\section{Motion in regular sea waves}

\section{Irregular sea}

\section{Motion in irregular sea waves}

\section{Simulation of motion in irregular sea}


