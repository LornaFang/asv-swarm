\section{Introduction}
The marine environment in which an autonomous surface vehicle (ASV) operates is 
highly dynamic in nature. An ASV in sea is subjected to forces due to wave, wind 
and surface current. These forces vary with time in intensity and direction and 
affects the motion of the ASV. A simulator that simulates the motion of an ASV 
should be able to capture the dynamics of marine environment and its influence 
on the ASV's motion.

There are two approaches for estimating the response motion of a body in waves - 
time domain analysis, and frequency domain analysis. In time domain analysis 
force exerted by wave on the hull and the corresponding vessel response motion 
is calculated for each instance of time (or simulation step). The wave force is
estimated by integrating the wave pressure along the wetted surface of the hull.
The wetted surface is the hull surface below the water surface (or the wave
profile) and is obtained by finding the intersection of the hull geometry with
the geometry representing the wave profile. The response motion for each time
step is then obtained by solving the equations of motion of rigid body dynamics. 

In frequency domain analysis, unlike time domain analysis, there are two key
assumptions regarding the wave and the vessel response to the wave. The 
irregular wave is considered as a result of superposition of many regular waves
and the vessel response motion is considered as the resultant of the vessel's 
response to each individual regular component of the exciting force. It is also 
assumed that the vessel response motion varies linearly to the amplitude of the 
exciting wave force. These two assumptions makes it possible to split the entire
calculation procedure into two parts. The first part is to create a reference 
table of vessel response motion to regular waves of unit amplitude and varying
frequencies. In the second part the irregular wave is broken down into regular
component waves and the vessel response to each of the component wave is 
obtained by scaling the response from the table in the first part. The actual
vessel response is then obtained as resultant of the motion response to
individual component.

Time domain analysis is the most commonly used method for simulating vessel 
motion in waves but this method is resource intensive since the calculations for
finding the intersection of the 3D hull geometry with the irregular wave 
geometry and the integration of the wave pressure force along the wetted hull 
surface are to be performed for each time step. This is a disadvantage in cases 
where high real time to simulation time ratio is required. Frequency domain 
analysis, on the other hand, has an advantage with respect to simulation speed. 
This is mainly due to the fact that the complex, resource intensive part of 
calculating vessel response to regular wave components needs to be performed 
only once for a vessel and the data can be saved and referred during actual 
simulation time.   

Frequency domain analysis has some limitations. It is only suitable in the case
where the sea surface is linear superposition of regular waves and vessel motion
varies linearly to wave amplitude. 

TODO: Add example cases where frequency domain analysis is not suitable.


\section{Procedure for frequency domain analysis based simulation}
\subsection{The sea surface}
\subsection{Vessel response motion}

