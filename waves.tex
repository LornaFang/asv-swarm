\section{Ocean Waves} 

The outstanding visible characteristic of an open sea surface is its 
irregularity. The waves on the surface do not repeat periodically in time or 
space. Yet, over a wide area and for a period of time, the sea surface maintains
a characteristic appearance. Study on wave data have shown that even though the
sea surface is irregular, the wave elevation readings is Gaussian in nature and 
is statistically a constant for a given area for a certain period of time. It is
therefore possible to define a sea condition, for region for (a short) period of 
time, using statistical parameters such as mean elevation and variance. The mean
elevation will however be zero, since wave does not change the water level in
the sea. Which means, considering Gaussian distribution for wave elevations, the
sea condition can be defined using variance alone.

For the purpose of this research we consider waves generated due to storms, that
is waves that are generated by the interaction of wind and water surface. The
two main physical process involved in the generation of storm waves are the
friction between air and water and the local variation of pressure field due to
wind. Even though there are many processes that will affect the growth and
propagation of waves, for waves of small amplitude, it is primarily governed by
the principle of superposition. So if $\zeta_1(x,y,t)$ and $\zeta_2(x,y,t)$ are
two wave systems then $\zeta_1(x,y,t) + \zeta_2(x,y,t)$ is also a wave system.
Based on this assumption a sea surface can be defined as a linear superposition
of waves having different amplitude, frequency, wave length and direction. 
However, it should be noted that the assumption regarding the linear 
superimposability of waves fails when the wave system is too steep and wave 
breaking occurs.

\subsection{2D regular sea waves}

A regular sea wave is a harmonic waves with crests that are infinitely long,
parallel and equally spaced and having constant wave heights. The general 
equation of a regular long crested wave travelling at a angle $\mu$ to the 
$x$-axis is:
\begin{equation}
  \zeta (x,y,t) = \zeta_a \cos[k(x \cos \mu + y \sin \mu) - \omega t + \epsilon]
  \label {eq: 2D wave equation}
\end{equation}
Where $\epsilon$ is a phase angle. 

The wave system travels perpendicular to the line of crests with a velocity
$V_c$. It is assumed that water is incompressible and has zero viscosity. Based
on these assumptions the motion of water particle in the wave can be described
using a quantity called *velocity potential* which is defined as a function
whose negative derivative in any direction yields the velocity component of the
fluid in that direction. Given
below is a simplified equation for velocity potential:
\begin{equation}
  \phi = - \zeta_a V_c \frac{\cosh k(z + h)}{\sinh k h} \sin k(x - V_c t)
  \label {eq: 2D wave velocity potential}
\end{equation}
Where: \\ 
$\zeta_a$ is the wave amplitude at the water surface\\ 
$k = \frac{2 \pi}{L_w}$; $L_w$ is the wave length\\ 
$z$ is the vertical distance of the water particle from the surface and is 
measured negative in the downward direction\\ 
$h$ is the water depth (distance from sea surface to seabed)\\ 
$V_c$ is the wave velocity (or celerity)\\ 
$t$ is time\\ 
$x$ is the x-coordinate of the water particle\\
Figure \ref{fig: RegularWave-1} shows the propagation of wave defined in 
equation \ref{eq: 2D wave velocity potential}.
\begin{figure} 
  \missingfigure{Bhattacharyya Figure 3.13} 
  \caption{Propagation of 2D regular sea wave} 
  \label{fig: RegularWave-1} 
\end{figure}

For deep water, ie. where $h \gg \frac{L_w}{2}$, 
\begin{equation}
  \frac{\cosh k(z + h)}{\sinh k h} \approx e^{k z}
  \label{eq: ratio approx for deep water}
\end{equation}

Substituting equation \ref{eq: ratio approx for deep water} in equation 
\ref{eq: 2D wave velocity potential} we get:
\begin{equation}
  \phi = - \zeta_a V_c e^{k z} \sin k(x - V_c t)
  \label{eq: 2D wave velocity potential for deep water}
\end{equation}

The wave causes variation in the distribution of pressure below the water
surface and the equation for variation of pressure head due to waves is:
\begin{equation}
  \zeta = \frac{k \zeta_a {V_c}^2}{g} \frac{\cosh k(z + h)}{\sinh k h} \cos k(x
  - V_c t)
  \label{eq: pressure head variation}
\end{equation}

For deep water the above equation can be approximated as:
\begin{equation}
  \zeta = \zeta_a e^{k z} \cos k(x - V_c t)
  \label{eq: pressure head variation for deep water}
\end{equation}

For simple harmonic motion: 
\begin{equation}
  \omega = \frac{2 \pi}{T_w} = k V_c
  \label{eq: simple harmonic motion}
\end{equation}
where: \\
$T_w$ is the time period of the simple harmonic wave\\
$\omega$ is the circular frequency of the simple harmonic wave\\

Substituting equation \ref{eq: simple harmonic motion} in equations  
\ref{eq: 2D wave velocity potential} to
\ref{eq: pressure head variation for deep water}:\\
Velocity potential for any water depth: 
\begin{equation}
  \phi = - \zeta_a V_c \frac{\cosh k(z + h)}{\sinh k h} \sin (kx - \omega t)
  \label {eq: 2D wave velocity potential wrt frequency}
\end{equation}
Velocity potential for deep water: 
\begin{equation}
  \phi = - \zeta_a V_c e^{k z} \sin (k x - \omega t)
  \label {eq: 2D wave velocity potential wrt frequency deep water}
\end{equation}
Pressure head variation due to wave for any water depth:
\begin{equation}
  \zeta = \frac{k \zeta_a {V_c}^2}{g} \frac{\cosh k(z + h)}{\sinh kh} \cos (kx
  - \omega t)
  \label{eq: pressure head variation wrt frequency}
\end{equation}
Pressure head variation due to wave for deep water:
\begin{equation}
  \zeta = \zeta_a e^{k z} \cos (k x - \omega t)
  \label{eq: pressure head variation wrt frequency deep water}
\end{equation}

The pressure at any point is given by:
\begin{equation}
  p = \rho g (-z + \zeta)
\end{equation}
\begin{equation}
  p = - \rho g z + \frac{k \zeta_a {V_c}^2}{g} \frac{\cosh k(z + h)}{\sinh k h} 
  \cos (kx - \omega t)
  \label{eq: pressure variation}
\end{equation}

The total wave energy per unit area is:
\begin{equation}
  E = \frac{1}{2} \rho g {\zeta_a}^2
  \label{eq: energy per unit area}
\end{equation}

The variance, or mean-square value of surface elevation as a function of time
is:
\begin{equation}
  S = 
  \langle\zeta (t)^2 \rangle = 
  \lim_{T \to \infty} 
  \frac{1}{T} 
  \int\limits_{-\frac{T}{2}}^{\frac{T}{2}} 
  \zeta^2(t) dt
  \label{eq: variance}
\end{equation}
For simple harmonic motion of frequency $\omega$, the above equation for 
variance can be written as:
\begin{equation}
  S(\omega) = 
  \langle\zeta (t)^2 \rangle = 
  \frac{1}{2} {\zeta_a}^2
  \label{eq: variance for simple harmonic}
\end{equation}

Based on equation \ref{eq: variance for simple harmonic}, equation 
\ref{eq: energy per unit area} can be written as:
\begin{equation}
  E = \rho g S(\omega)
\end{equation}


\section{Irregular sea}

Oceanographers have found that irregular sea can be described by statistical
mathematics on the basis of the assumption that a large number of regular waves
having different wave lengths, direction, frequency and amplitude are linearly
superimposed. It is convenient to begin with a simple case of wave pattern
observed at a single point (ie. $x = y = 0$) and assuming that all component
waves are travelling in the same direction ($\mu = 0$). Based on equation 
\ref{eq: 2D wave equation}, the simplified equation of a compound wave (ie. a 
wave that consist of many component waves) is:
\begin{equation}
  \zeta(t) = \sum _{i} (\zeta_a)_i \cos(-\omega_i t + \epsilon_i)
  \label{eq: 2D irregular wave equation}
\end{equation}

It is convenient to define wave components in terms of a function called 
\textit{variance spectrum, $S(\omega)$}. For any particular wave frequency,
$\omega_i$, the variance of the wave amplitudes for a narrow band of frequency,
$\delta \omega$, centred on $\omega_i$ is given by:
\begin{equation}
  \langle \zeta_i (t)^2 \rangle = S(\omega_i) \delta \omega
  \label{eq: variance of a narrow band}
\end{equation}
As $\delta \omega \to 0$, it becomes a simple harmonic wave. The mean value for
a wave elevations for a simple harmonic wave is $0$ (wave does not increase the
mean water level) and the variance is given by:
\begin{equation}
  \langle \zeta_i (t)^2 \rangle = \frac{1}{2} {(\zeta_a)_i }^2
\end{equation}
Applying this is equation \ref{eq: vaiance of a narrow band}, we get:
\begin{equation}
  \frac{1}{2} {(\zeta_a)_i}^2 = S(\omega_i) \delta \omega
\end{equation}
\begin{equation}
  (\zeta_a)_i = \sqrt{2 S(\omega_i) \delta \omega}
\end{equation}


\section{Motion in regular sea waves}

\section{Motion in irregular sea waves}

\section{Simulation of motion in irregular sea}


