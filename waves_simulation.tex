\section{Simulation of ocean waves} \label{Simulation of ocean waves}

This section presents a rough structure of the classes and how the data are 
grouped under each class. It also captures the main functions that will be
called when simulating the ocean waves. The aim of this documentation is only to
provide a rough overview and does not aim to mirror the actual implementation. 
The source code documentation will contain the most accurate and in-depth 
description of the classes and its members. 

The ocean is represented by a surface. The contour of the surface is controlled
by an $n \times n$ matrix of points called the \textit{control points}· The sea
surface is represented by the class \textit{SeaSurface} (section 
\ref{Class SeaSurface}) and the control points are represented by the class
\textit{SeaSurfaceControlPoint} (section \ref{Class SeaSurfaceControlPoint}). 
Random waves are generated on the surface by the interaction of several regular 
waves which are represented by the class \textit{RegularWave} (section 
\ref{Class RegularWave}). The statistically property of the random sea generated 
through the regular waves is controlled by a directional spectrum, which is 
represented by the class \textit{DirectionalSpectrum} (section 
\ref{Class DirectionalSpectrum}). The directional spectrum is a collection of 
point spectrum for each direction ranging from $\sfrac{-\pi}{2}$ to 
$\sfrac{\pi}{2}$.  The directional spectrum contains a table where each row 
represents a point spectrum for a direction and the columns of the table relate 
to each frequency.

\subsection{Steps for simulating the random sea} 
Following are the main steps to initialise the sea surface to a wave spectrum
and start simulation.
\label{Process for simulating the random sea}
\begin{itemize}
  \item Create an object of the class \textit{SeaSurface}.
	\item Initialise the seas surface to a wave spectrum by calling 
    \textit{SeaSurface.Init()}. A call to the method \textit{SeaSurface.Init()} 
    will also generate and initialise all the regular waves in the sea surface.
  \item Start simulation by calling \texit{SeaSurface.Simulate()}.
\end{itemize}

\subsection{Class SeaSurface} \label{Class SeaSurface}

The class \textit{SeaSurface} is used to define a square patch of the sea 
surface.

\textbf{Member variables in the class:}
\begin{itemize}
  \item seaPatchSize - size of the square sea patch represented by 
    \textit{SeaSurface}.
  \item seaSurfaceControPoints - an array of points to control the contour of 
    the sea surface.
  \item directionalSpectrum - contains a table of variance values for each
    direction and frequency.
  \item regularWaves - array of all regular waves on the sea surface.
\end{itemize}

\textbf{Methods in the class:}
\begin{itemize}
  \item SetCtrlPointMatSize() - sets the number of control points on the
    surface.
  \item Init()\\
    $>$ Initialise the directional spectrum.\\
    $>$ Create the regular waves in the sea as described below:\\
    $>>$ for each row in directional spectrum (ie: direction):\\
    $>>>$ for each column in directional spectrum (ie: frequency):\\
  	$>>>>$ Get the variance value from the directional spectrum.\\
  	$>>>>$ Calculate the amplitude from variance.\\
    $>>>>$ Generate a random value for wave phase.\\
    $>>>>$ Create a regular wave with the amplitude, phase, frequency and 
          direction.\\
    $>>>>$ Insert the regular wave created into the array regularWaves.
  \item Simulate() \\
    $>$ Initialise time = 0.\\ 
  	$>$ for each point in the array seaSurfaceControlPoints:\\
  	$>>$ for each wave in array regularWaves:\\
  	$>>>$ Calculate elevation due to each wave.\\
    $>>$ Sum the elevations for all waves.\\
    $>>$ Set SeaSurfaceControlPoint.z = sum of elevations.\\
  	$>$ Increment time and continue above loop.
\end{itemize}

\subsection{Class DirectionalSpectrum} \label{Class DirectionalSpectrum}
Class to generate a directional spectrum.

\textbf{Member variable in the class:} 
\begin{itemize} 
  \item countDirections - the number of directions for which point spectrum data 
    will be generated.  
  \item countFrequencies - the number of frequencies bands that will be used for 
    each point spectrum.  
  \item directions - array containing direction angles.
  \item frequencies - array containing the frequency intervals.
  \item spectrum - table containing variance values. Each row corresponds to 
    a direction and each column corresponds to a frequency band.  
\end{itemize}

\textbf{Methods in the class:}
\begin{itemize}
	\item SetCountDirections() - sets value for member variable countDirections.
	\item SetCountFrequencies() - sets value for member variable countFrequencies.
	\item Init()\\
    $>$ Set values for the array directions. \\
		$>$ Set values for the array frequencies. \\
		$>$ For each angle:\\
		$>>$ For each frequency:\\
		$>>>$ Calculate variance $S(\omega)G(\mu)$\\
    $>>>$ Enter the variance into the table spectrum. 
\end{itemize}

\subsection{Class RegularWave} \label{Class RegularWave}
Class to represent a regular wave.

\textbf{Member variables in the class:}
\begin{itemize}
	\item direction - direction of travel of the wave.
	\item phase - the random generated phase value of the wave.
	\item frequency - the circular frequency of the regular wave.
	\item amplitude - the amplitude of the regular wave.
\end{itemize}

\subsection{Class SeaSurfaceControlPoint} \label{Class SeaSurfaceControlPoint}
Class to contain the geographic coordinates of a control point on the sea 
surface. 

\textbf{Member variables in the class:} 
\begin{itemize} 
  \item x - the x coordinate 
  \item y - the y coordinate
  \item z - the z coordinate
\end{itemize}

